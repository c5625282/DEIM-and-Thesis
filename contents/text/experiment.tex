\section{実験}\label{sec:experiment}
\subsection{実験設定}
実験の前処理として,全特徴量を標準化し,この標準化空間において欠損の導入および補完を行った
.完全データに対し,セル単位のMCAR(Missing Completely At Random)で欠損を導入し,
欠損率 $r_{miss} \in \{0.1,0.3,0.5\}$ とした.欠損セル数は $[Ndr_{miss}]$ ($N$ はサンプル数,
$d$ は特徴量数)であり,
位置は一様無作為に重複なしで選択した.また,各反復において全手法で同一の欠損位置を用いることで
公平性を担保した.各欠損率につき50回の独立反復を実施し,毎反復で欠損位置を再サンプリングした.
評価はすべて標準化空間において行い,欠損セルのみを対象とした平均二乗誤差(MSE)を指標とした.
表1は各手法・各欠損率における50回反復の平均(標準偏差)である.

\subsection{実験に用いたデータ}
実験には,UCI Machine Learning Repositoryにて公開されているベンチマークデータを利用した.採用したデータセットは,“iris”,“wine”,“diabetes”である.ここでは上記のデータに対してクラスとして扱われるカテゴリカル変数を削除し,特徴量のみを用いて欠損補完精度を評価した.

\subsection{実験結果}
実験結果は表1に示すとおりである.表中の数値は各手法・各欠損率における50回反復のMSEの平均を表している.また,括弧内の数値は試行毎のMSEの標準偏差を示している.

irisでは,欠損率が0.1および 0.3でmissForestが最良な結果を示し,提案手法も同等の精度を示した.一方,欠損率が0.5ではkNNが最小となり,低次元データではユークリッド距離に基づく近傍探索が有効に働いたと考えられる.このことから,単純なデータ構造に対してはkNNやmissForestと同等の性能を発揮しつつ,提案手法は過剰適応することなく安定した補完が可能であることが確認された.
wineでは,全欠損率で提案手法が最小誤差となり,優位性が示された.これは特徴量間の相関が比較的強い中規模データでは,特徴選択とOOB重みによるアンサンブル効果が有効に機能した結果と考えられる.

diabetesでは欠損率0.1の際にmissForestが優れたが,0.3と0.5では提案手法が最小誤差を示し,中程度の相関を持つデータに対しても有効性を確認できた.

% \begin{table}[h]
%     \centering
%     \caption{提案法(RF-Ens)と既存補完手法の精度比較}
%     \begin{|c|c|c|c|c|c|}
%     \hline
%     data set \multicolumn{1}{c|}{missing rate} & \multicolumn{1}{c|}{mean} & \multicolumn{1}{c|}{kNN(k=5)} & \multicolumn{1}{c|}{missForest} & \multicolumn{1}{c|}{RF-Ens (提案)} & \multicolumn{1}{c|}{RF-Ens (proposed)} \\ \hline \hline

%     \end{|c|c|c|c|c|}    

\begin{table}[h]
\centering
\footnotesize
\caption{提案法(RF-Ens)と既存補完手法の精度比較}
\label{tab:comparison}

\begin{tabularx}{0.98\linewidth}{|c|c|c|c|c|c|}
\toprule
\makecell{data\\set} & \makecell{missing\\rate} 
    & mean & kNN($k{=}5$) & missForest & RF-Ens \\
\midrule

\multirow{3}{*}{iris}
 & 0.1 & 1.022(0.017) & 0.261(0.012) & 0.235(0.014) & 0.243(0.010) \\
 & 0.3 & 1.018(0.008) & 0.430(0.011) & 0.365(0.014) & 0.369(0.010) \\
 & 0.5 & 1.023(0.001) & 0.545(0.009) & 0.609(0.016) & 0.548(0.012) \\
\midrule

\multirow{3}{*}{wine}
 & 0.1 & 1.003(0.013) & 0.528(0.011) & 0.491(0.010) & 0.488(0.010) \\
 & 0.3 & 1.014(0.007) & 0.634(0.006) & 0.591(0.006) & 0.583(0.006) \\
 & 0.5 & 1.021(0.004) & 0.812(0.005) & 0.786(0.009) & 0.726(0.005) \\
\midrule

\multirow{3}{*}{diabetes}
 & 0.1 & 1.016(0.010) & 0.643(0.007) & 0.510(0.006) & 0.518(0.006) \\
 & 0.3 & 1.010(0.004) & 0.834(0.004) & 0.649(0.005) & 0.632(0.004) \\
 & 0.5 & 1.008(0.003) & 1.007(0.003) & 0.855(0.005) & 0.779(0.004) \\
\bottomrule
\end{tabularx}

\end{table}





