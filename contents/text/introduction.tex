\section{はじめに}
近年,センシング技術や情報通信の発達により,医療,アンケート調査,Iotセンサ,顧客属性や
販売実績といったビジネスデータなど,様々な分野において大量かつ多量なデータを収集できるよ
うになっている.しかし,現実の計測環境では常に完全なデータが得られるとは限らず,計測機器
の誤作動や通信エラー,人為的な誤入力などにより,データ中に欠損値が含まれる状況がしばしば
発生する.このような欠損を含むデータをそのまま用いて分析や予測を行うと,解析精度の低下や
推定値の偏り,サンプルサイズの減少による識別精度の低下といった問題を引き起こす.そのため,
欠損値を適切に補完し,元の情報をできる限り維持した上で活用することは,信頼性のあるデータ
分析を支える上で不可欠な前処理である.

基本的な欠損値補完手法としては,平均値補完やホットデック法などの単一代入法が広く用いられ
ているが,これらはデータの背後に隠れるクラスタ構造や非線形性を考慮できない,またはサンプル
のばらつきに強く影響されるといった課題を抱えており,補完結果の頑健性や安定性に限界がある.
一方,多重代入法は不確実性を考慮できる理論的に優れた手法であるが,モデル選択に強く依存する
という課題がある.その他に,機械学習に基づく手法としてk近傍法やランダムフォレストを用いた
missForestがある.missForestは非線形性や特徴量間の相互作用をある程度捉えることが可能で
あるが,すべての特徴量を一律に用いるため,ノイズ的な特徴量や寄与の小さい特徴量が学習に
含まれ,分割の質が下がり,結果として予測精度が低下する場合がある.また,欠損率が高い場合や
サンプル数に比べて特徴量が多い場合,過学習や性能劣化が生じやすい.

そこで本研究では,複数の特徴選択基準に基づいて有効な特徴量を抽出し,各基準で構築した
ランダムフォレストの予測をOOB $R^2$ に応じて加重平均するアンサンブル補完手法を提案する.