\section{まとめと考察}
iris データセットのような小規模なデータセットでは,提案手法は既存法と同等の精度を維持した.一方で,中規模以上のデータセットにおいては,特徴量選択を導入することで欠損補完の精度改善が確認された.これは,ノイズ的あるいは寄与の小さい特徴量を含めずに学習することで,欠損位置の違いによる推定値のばらつきが抑制され,より安定した補完が可能となったためである.さらに,相関の強い特徴量を優先的に利用することで,欠損値を説明力の高い情報に基づいて推定できるようになった.このように,特徴量選択はモデルの安定性と汎化性能を高め,欠損補完における頑健性の向上に寄与したと考えられる.
